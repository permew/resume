% !TEX TS-program = xelatex
% !TEX encoding = UTF-8 Unicode
% !Mode:: "TeX:UTF-8"

\documentclass{resume}
\usepackage{zh_CN-Adobefonts_external} % Simplified Chinese Support using external fonts (./fonts/zh_CN-Adobe/)
% \usepackage{NotoSansSC_external}
% \usepackage{NotoSerifCJKsc_external}
% \usepackage{zh_CN-Adobefonts_internal} % Simplified Chinese Support using system fonts
\usepackage{linespacing_fix} % disable extra space before next section
\usepackage{cite}

\begin{document}
\pagenumbering{gobble} % suppress displaying page number

\name{王晓曦}

\basicInfo{
  \email{xiaoximew@icloud.com} \textperiodcentered\ 
  \phone{(+86) 176-110-17922} \textperiodcentered\ 
  }
 
\section{\faGraduationCap\  教育背景}
\datedsubsection{\textbf{北京工业大学}, 北京, 中国}{2018 -- 至今}
\textit{在读硕士研究生}\ 生物医学工程, 预计 2021 年 6 月毕业
\datedsubsection{\textbf{北京工业大学}, 北京, 中国}{2014 -- 2018}
\textit{学士}\ 信息安全

\section{\faUsers\ 实习/项目经历}
\datedsubsection{\textbf{领智云公司} 北京{2019年1月 -- 至今}
\role{后端开发实习}
\textbf{Keywords}: Python, Django, Flask, MySQL, MongoDB, OpenTSDB, Redis
\begin{itemize}
  \item 操作日志API的开发
  \item 设置虚拟服务器以测试管理系统
  \item 改进了原有一些管理系统的可读性和性能
\end{itemize}

\datedsubsection{\textbf{国鼎科技有限公司} 北京}{2017.6 -- 2017.9}
\role{数据分析实习}{}
\textbf{Keywords}: 用户画像,大数据
\begin{itemize}
  \item 通过授权的用户信息(设备UID,在线时间等)构建用户角色
  \item 部署了从数据收集,数据分析到用户角色的管道
\end{itemize}

\datedsubsection{基因数据的整合分析 \textbf{个人项目} }{2018.9 -- 至今}
\textbf{Keywords}: 数据分析, 多维数据, 机器学习, Excel, SPSS, R, Python
\begin{itemize}
  \item 应用多种机器学习算法,包括SVM,逻辑回归,随机森林,弹性网络,神经网络,通过组织样本数据预测癌症(预处理的TCGA数据集)
  \item 导入GTEx数据集平衡正样本和负样本
  \item 使用GEO数据集作为评估的验证集
  \item 早期甲状腺癌预测模型的准确率达到99%
\end{itemize}

\datedsubsection{通用论坛正文提取 \textbf{小组项目} }{2017}
\datedline{\textit{\nth{1} Prize}, 第五届泰迪杯数据挖掘挑战赛 }{}
\textbf{Keywords}: Python, 爬虫, 数据分析
\begin{itemize}
  \item 构建了一个Python爬虫来获取在线论坛文本数据
  \item 参与了项目的几个阶段,包括论坛数据分析,提取算法设计和测试
\end{itemize}

\datedsubsection{Cool 编译系统的设计与实现 \textbf{个人项目}}{2017}
\datedline{\textit{特优毕业设计}}{}
\textbf{Keywords}: C++, 编译器设计
\begin{itemize}
  \item 实现了一些语言特性,包括面向对象,静态类型和自动内存管理
  \item 实现了编译器的多个关键部分,包括词法分析器,解析器,语义分析程序和代码生成器
\end{itemize}

% Reference Test
%\datedsubsection{\textbf{Paper Title\cite{zaharia2012resilient}}}{May. 2015}
%An xxx optimized for xxx\cite{verma2015large}
%\begin{itemize}
%  \item main contribution
%\end{itemize}
\section{\faCogs\ 技能}
\begin{itemize}[parsep=0.5ex]
  \item 丰富的数据分析和服务器端开发经验
  \item 办公软件: Excel, Word, PowerPoint
  \item 程序设计语言: Python > C == C++ \ | \ 平台: Linux, MacOS
\end{itemize}

%% Reference
%\newpage
%\bibliographystyle{IEEETran}
%\bibliography{mycite}
\end{document}
